\begin{center}\section*{PRGSY Robot }\end{center} 

\begin{center}\end{center}  \begin{DoxyAuthor}{Autor}
David Malgiaritta 

Sascha Waser \par
\par
\par

\end{DoxyAuthor}
Im Modul Systemnahes Programmieren an der Hochschule f\"{u}r Technik \& Architektur Luzern wird w\"{a}hrend eines Semesters ein spezielles Projekt durchgef\"{u}hrt. Ziel des Projektes ist es, anhand einer industrienahen Hard-\/ und Software Plattform, das .NET Framework und insbesondere die Programmiersprache C\# kennenzulernen. \par
 Der Code wird auf Windows XP/Vista/7 und Visual Studio 2008 entwickelt. Der Roboter hingegen wird mit Windows CE und dem .NET Compact Framework betrieben. Diese plattform unabh\"{a}ngige Software-\/Entwicklung stellt eine besondere Herausforderung an die Studierenden.

\par
\subsection*{Technische Daten }

\subsubsection*{Antrieb }


\begin{DoxyPre}
 Getriebe-\"{U}bersetzung                     14 : 1
 Encoder-Perioden pro Motor-Umdrehung     512 
 Ticks pro Encoder-Periode                4 
 Ticks pro Rad-Umdrehung                  28'672 
 Rad-Durchmesser                          76 mm 
 Rad-Abstand (Achsl\"{a}nge)                  257 mm
 \end{DoxyPre}
 \subsubsection*{Motorsteuerung }


\begin{DoxyPre}
 Motor-IC                                 LM629 (8MHz-Clock)
 LM629 Sample Intervall                   256us
 Leistungsstufe                           L6206
 \end{DoxyPre}


 